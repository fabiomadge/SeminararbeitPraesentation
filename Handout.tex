\documentclass[a4paper]{article}
\usepackage[ngerman]{babel}
\usepackage[utf8]{inputenc}
\usepackage{amsmath,amsfonts,amssymb}
\usepackage{graphicx}
\usepackage{pstricks}
\usepackage{pst-plot,pst-math}
\usepackage[left=1.5cm, right=1.5cm, top=1.0cm, bottom=1.0cm]{geometry}
\pagenumbering{gobble}
\title{Einführung in die Kegelschnitte}
\author{Fabio Madge Pimentel}
\date{December 19, 2013}

\begin{document}

\maketitle

Die Kegelschnitte werden durch den Schnittrand eines Doppelkegels und einer Ebene gebildet.

\vspace{4 mm}

\begin{table}[ht]
\begin{minipage}[b]{0.45\linewidth}
\centering

	\centering
		\resizebox{.9\linewidth}{!}{
					\begin{pspicture}(-6,-4)(6,4)
	\psgrid[griddots=3, subgriddiv=0]
	\psellipse[linewidth=2pt](0, 0)(5.035,3.035)
	\uput*[270](0,0){$M$}
	\uput*[310](2.5,-2.6){$P$}
	\uput*[60](4,0){${F}_1$}
	\uput*[180](-4,0){${F}_2$}
	\uput*[0](5,0){${S}_1$}
	\uput*[180](-5,0){${S}_2$}
	\uput*[90](0,3){${S}_3$}
	\uput*[270](0,-3){${S}_4$}
	\uput*[270](2.5,0){$a$}
	\uput*[180](0,1.5){$b$}
	\uput*[90](-2,0){$e$}
	\rput[b]{36.87}(-2,1.5){\uput[d](0,0){$a$}}
	\rput[b]{-36.87}(2,1.5){\uput[u](0,0){$a$}}
	\psline(0,0)(5,0)%a
	\psline(0,0)(0,3)%b
	\psline(0,0)(-4,0)%e
	\psline(0,3)(4,0)%S3F1
	\psline(0,3)(-4,0)%S3F2
	\psline(2.5,-2.6)(4,0)%PF1
	\psline(2.5,-2.6)(-4,0)%PF2
	\psdots[linecolor=blue](0,0)%M
	\psdots[linecolor=blue](2.5,-2.6)%P
	\psdots[linecolor=blue](4,0)%F1
	\psdots[linecolor=blue](-4,0)%F2
	\psdots[linecolor=blue](5,0)%S1
	\psdots[linecolor=blue](-5,0)%S2
	\psdots[linecolor=blue](0,3)%S3
	\psdots[linecolor=blue](0,-3)%S4
\end{pspicture}
		}

\begin{displaymath}
	\begin{array}{rcl}
		Ellipse  & = & \{P\in E^2\;|\;\overline{{PF}_1} + \overline{{PF}_2} = 2a\}
	\end{array}
\end{displaymath}
\begin{displaymath}
	\begin{array}{rcl}
		\frac{x^2}{a^2} + \frac{y^2}{b^2} &=& 1
	\end{array}
\end{displaymath}
\end{minipage}
\hspace{0.5cm}
\begin{minipage}[b]{0.45\linewidth}
\centering

	\centering
		\resizebox{.9\linewidth}{!}{
					\begin{pspicture}(-3,-4)(5,4)
	\psgrid[griddots=3, subgriddiv=0]
	\rput{90}(4,-4){\parabola[linewidth=2pt](0,0)(4,4)}
	\pscircle[linestyle=dashed](1,0){3}
	\uput*[143](0,0){$S$}
	\uput*[0](1,0){$F$}
	\uput*[180](-1,0){$F'$}
	\uput*[0](2,0){$D$}
	\uput[240](2,-2.83){$P$}
	\uput{.3}[245](2,2.83){$P'$}
	\uput*[180](-1,3.5){$l$}
	\uput*[180](2,3.5){$d$}
	\uput[40](0,0){$p$}
	\psline(-1,-4)(-1,4)%l
	\psline[linestyle=dashed](2,-4)(2,4)%d
	\psline(-1,0)(1,0)%p
	\psline(1,0)(2,-2.83)%PF
	\psline(-1,-2.83)(2,-2.83)%Pl
	\psdots[linecolor=blue](0,0)%S
	\psdots[linecolor=blue](1,0)%F
	\psdots[linecolor=blue](-1,0)%F'
	\psdots[linecolor=blue](2,0)%D
	\psdots[linecolor=blue](2,-2.83)%P
	\psdots[linecolor=blue](2,2.83)%P'
\end{pspicture}
		}

\begin{displaymath}
	\begin{array}{rcl}
		Parabel  & = & \{P\in E^2\;|\;\overline{PF} = \overline{Pl}\}\
	\end{array}
\end{displaymath}
\begin{displaymath}
	\begin{array}{rcl}
		y^2 &=& 2px
	\end{array}
\end{displaymath}
\end{minipage}
\end{table}

\begin{figure}[h]
	\centering
		\resizebox{.46\linewidth}{!}{
			\begin{pspicture}(-5,-4)(5,4)
	\psset{algebraic,plotpoints=500}
	\psgrid[griddots=3, subgriddiv=0]
	\psplot[linewidth=2pt]{1}{4.123}{sqrt(x^2-1)}
	\psplot[linewidth=2pt]{1}{4.123}{-sqrt(x^2-1)}
	\psplot[linewidth=2pt]{-1}{-4.123}{sqrt(x^2-1)}
	\psplot[linewidth=2pt]{-1}{-4.123}{-sqrt(x^2-1)}
	\pscircle[linestyle=dashed](1.4142,0){1}
	\pscircle[linestyle=dashed](-1.4142,0){1}
	\pscircle[linestyle=dashed](1.4142,0){3}
	\pscircle[linestyle=dashed](-1.4142,0){3}
	\uput*[270](0,0){$M$}
	\uput{0.3}[20](1.4142,1){$P$}
	\uput{0.3}[340](1.4142,-1){$P'$}
	\uput{0.3}[160](-1.4142,1){$P''$}
	\uput{0.3}[200](-1.4142,-1){$P'''$}
	\uput*{0.1}[0](1.4142,0){${F}_1$}
	\uput*[180](-1.4142,0){${F}_2$}
	\uput[240](1,0){${S}_1$}
	\uput[300](-1,0){${S}_2$}
	\uput*[0](2,0){$D$}
	\uput*[90](0.5,0){$a$}
	\uput*[150](0,0.5){$b$}
	\uput*{0.04}[90](-0.7071,0){$e$}
	\psline(0,0)(1,0)%a
	\psline(0,0)(0,1)%b
	\psline(0,0)(-1.4142,0)%e
	\psline(1.4142,1)(1.4142,0)%PF1
	\psline(1.4142,1)(-1.4142,0)%PF2
	\psdots[linecolor=blue](0,0)%M
	\psdots[linecolor=blue](1,0)%S1
	\psdots[linecolor=blue](-1,0)%S2
	\psdots[linecolor=blue](1.4142,0)%F1
	\psdots[linecolor=blue](-1.4142,0)%F2
	\psdots[linecolor=blue](2,0)%D
	\psdots[linecolor=blue](1.4142,1)%P
	\psdots[linecolor=blue](1.4142,-1)%P'
	\psdots[linecolor=blue](-1.4142,1)%P''
	\psdots[linecolor=blue](-1.4142,-1)%P'''
\end{pspicture}
		}
\end{figure}
\begin{displaymath}
	\begin{array}{rcl}
		Hyperbel & = & \{P\in E^2\;|\;|\overline{{PF}_2} - \overline{{PF}_1}| = 2a\}
	\end{array}
\end{displaymath}
\begin{displaymath}
	\begin{array}{rcl}
		\frac{x^2}{a^2} - \frac{y^2}{b^2} &=& 1
	\end{array}
\end{displaymath}	

\large{Lineare Exzentrität}

\begin{displaymath}
	\begin{array}{rcl}
		e & = & \sqrt{a^2\pm b^2}
	\end{array}
\end{displaymath}

\large{Numerische Exzentrizität}

\begin{displaymath}
	\epsilon = \frac{e}{a}
\end{displaymath}

\end{document}